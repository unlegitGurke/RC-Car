\subsection{Kern 0 / Sensoren und Kommunikation}
	Die Hauptaufgabe von Kern 0 ist es, mit dem Bordcomputer zu kommuniziern, alle Sensordaten auszulesen, und Komponenten, wie zum Beispiel L�fter, zu steuern.
	Die Seriellen Schnittstellen vom ESP32 und dem Bordcomputer sind mithilfe des Shields verbunden, um eine fl�ssige Kommunikation zu gew�hrleisten, ist ein Protokoll n�tig:
	
	\subsubsection{Aufbau von Paketen zwischen ESP32 und Bordcomputer}
		Der grunds�tzliche Aufbau eines Paktes zwischen ESP32 und Bordcomputer ist immer gleich, er besteht aus folgenden Teilen
			
			\begin{enumerate}
			
				\item Datentyp: LED, Temp, Volt, IMU1, IMU2, Sonar, FAN, Error //Alle m�glichen Daten, die zwischen ESP und Bordcomputer gesendet werden k�nnen. 
				
				\item Anzahl der Datenwerte
				
				\item Access, Read / Write / Both  //Sollen die Daten nur eingelesen werden, wird nur eine Antwort erwartet oder wird beides erwartet?
				
				\item Daten, je nach Anzahl der Datenwerte k�nnen hier beliebig viele Werte hintereinander geschrieben werden
				
				\item Error, wird ein Fehler gemeldet?
			
			\end{enumerate}
			
			Jedes Datenpaket beginnt mit einem Marker und endet mit einem Marker, in diesem Fall wurde als StartMarker "x" verwendet und als EndMarker "y". Die Daten sind in dazwischen in genannter Reihenfolge mit Beistrichen getrennt.
			Die Gr��e der verschiedenen Datenpaktet ist wiefolgt:
			
			\begin{table}[H]
				\centering
				\caption{Datenpakete}
				
				\begin{tabular}{|c|c|c|c|c|c|}
					
					\hline
					\multicolumn{6}{|c|}{\textbf{Datenpakete}}\\
					\hline
					\hline
					\textbf{Name} & \textbf{Datentyp} & \textbf{Anzahl der Variablen} & \textbf{Access} & \textbf{Daten} & \textbf{Error} \\
					\hline
					IMU1 & 1 & 6 & 1 & - & 0 \\
					\hline
					IMU2 & 2 & 3 & 1 & - & 0 \\
					\hline
					OctoSonar & 3 & 16 & 1 & - & 0 \\
					\hline
					Spannungssensoren & 4 & 5 & 1 & - & 0 \\
					\hline
					Temperatursensoren & 5 & 6 & 1 & - & 0 \\
					\hline
					L�fter & 6 & 4 & 2 & - & 0 \\
					\hline
					LEDs & 7 & 4 & 2 & - & 0 \\
					\hline
					IsOK & 0 & 0 & 2 & - & 0 \\
					\hline
					
				\end{tabular}
			\end{table}
			
			Access = 0 bedeutet, dass Daten nur vom ESP gelesen werden und es keine Antwort gibt.\\
			Access = 1 bedeutet, dass Daten an den Bordcomputer vom ESP geschickt werden sollen.\\
			Access = 2 bedeutet, dass Daten vom ESP eingelesen werden sollen und dass eine Antwort geschickt werden soll.\\
			\\
			Error = 0 bedeutet, das kein Fehler vorhanden ist.\\
			Error = 1 bedeutet, dass eine Warnung vorhanden ist.\\
			Error = 2 bedeutet, dass ein fataler Fehler vorhanden ist.\\
			
		\subsubsection{Einlesen der Pakete am ESP32}
			F�r die Kommunikation zwischen ESP32 und Bordcomputer ist eine eigene Bibliothek verantwortlich, so muss im Hauptprogramm nur eine Objekt "Computer" erstellt werden, um Daten einzulesen beziehungsweise zu empfangen.
			Die erste Funktion der Bilbiothek lie�t die Daten ein und speichert sie in einem Buffer:
			
			\lstset{style=VS2017}
			\begin{lstlisting}[language=C, caption={Einlesen eines Pakets}]
bool LattePandacomms::refresh(char startMarker, char endMarker) {
  
  _inProgress = false;
  
  while (Serial.available() > 0) {    //Checks for Serial Data
    
    byte x = Serial.read();   //Reads Serial Data
    
    if (x == startMarker && _inProgress == false) {    //First Byte = startMarker?
       
      _bytesRecvd = 0;      //Zero bytes have been recieved so far
      _inProgress = true;   //New Message has started
      
      for(int i = 0; i < sizeof(BufferIn); i++) {   //Clear BufferIn Array
        BufferIn[i] = '\0';
      }
      
    }
    
    if (_inProgress ) {   //Has the message started?
          
      if (_bytesRecvd<_maxMessage) {    //Max message limit?
        
        BufferIn[_bytesRecvd++] = x;  //Save current Byte to Buffer
        
      }
      
      if (x == endMarker) {     //End of message 
        
        _inProgress = false;    //Reset inprogress variable
        //allReceived = true;
        _nb = _bytesRecvd;
        return 1;
        
      }
      
    }

    
  }
  
  //Serial.println(BufferIn);   //DEBUG
  return 0;
  
}			
			\end{lstlisting}
			
			Diese Funktion wartet auf den StartMarker und speichert anschlie�end bis zum Erscheinen des EndMarkers jede Ziffer in ein Buffer-Array, welches danach weiterverarbeitet werden kann.
			
			\subsubsection{Dekodieren der Pakete am ESP32}
					Um das sich im Buffer befindliche Paket nun zu dekodieren sind zwei zus�tzliche Funktionen n�tig:\\
					Die Funktion \textit{bool removeMarkers()} entfernt den Start-und Endmarker, da diese nicht dekodiert werden m�ssen.\\
					
					\lstset{style=VS2017}
					\begin{lstlisting}[language=C, caption={RemoveMarkers()}]
bool removeMarkers(char* inputMessage,
                   char* outputMessage, 
                   char startMarker, 
                   char endMarker) {   
  
  int startIndex = 0;
  int endIndex = 0;

  for (int i = 0; i < strlen(inputMessage); i++) {
      if (inputMessage[i] == startMarker) {
        startIndex = i + 1;     // Skip startMarker
      } 
      else if (inputMessage[i] == endMarker) {
         endIndex = i;
         break;    // Stop once endMarker is found
      }
  }
  
	// Extract the substring between startMarker and endMarker into the output message
  if (endIndex > startIndex) {    
    strncpy(outputMessage, inputMessage + startIndex, endIndex - startIndex);
    outputMessage[endIndex - startIndex] = '\0';    // Null-terminate the output message
    return true;
  } 
  
  else {
    outputMessage[0] = '\0'; // Empty output message if 'q' is not found after 'x'
    return false;
  }  
  
}					
					\end{lstlisting}
					
					Mithilfe von der Funktion \textit{bool removeFirstValue()} ist m�glich den ersten Wert aus einem Char-Array zu l�schen. So kann nun immer der erste Wert eingelesen werden und dann gel�scht werden. Dies wird so oft widerholt, bis die gew�nschte Anzahl an Variablen(durch Anzahl der Variablen im Paket gegeben) eingelesen ist.\\
					
					\lstset{style=VS2017}
					\begin{lstlisting}[language=C, caption={RemoveMarkers()}]
bool removeFirstValue(char* inputMessage, char* outputMessage) { 
    
  int commaIndex = 0;
  
	//Search for first comma
  while (inputMessage[commaIndex] != ',' && inputMessage[commaIndex] != '\0') {   
    commaIndex++;
  }
  
	// If comma is found, copy the substring after first comma into output message
  if (inputMessage[commaIndex] == ',') {    
    strcpy(outputMessage, inputMessage + commaIndex + 1);
    return true;
  } 
  
  else {
    return false;
  }
  
}					
					\end{lstlisting}
					
					Diese eingelesen Variablen werden in Komponenten, die durch ein struct definiert sind, im public Teil der Class abgespeichert, so ist es m�glich vom Hauptcode auf diese zuzugreifen:
					
					\lstset{style=VS2017}
					\begin{lstlisting}[language=C, caption={Datenspeicherort und Struktur}]
struct Component {    //Struktur des Datenspeicherorts
  uint8_t Type;
  uint8_t nVal;
  uint8_t Access;
  uint8_t Data[maxnDataVar];
  uint8_t Error;
};

class LattePandacomms {
public:
  
  LattePandacomms();

  Component IsOk;		//Datenspeicher der einzelnen Komponenten
  Component IMU1;
  Component IMU2;
  Component Octosonar;
  Component Voltage;
  Component Temp;
  Component Fan;
  Component LED;
}						
					\end{lstlisting}
					
					Im Hauptprogramm k�nnen Daten eingelesen werden indem sie durch das Objekt aufgerufen werden (zum Beispiel: Objekt.IMU1.Data[1];).
			\subsubsection{Einlesen der Sensordaten}
					
					Zus�tzlich zur Kommunikation liest Kern 0 auch alle Sensordaten ein und steuert alle L�fter. Die Funktionen sind essenziell sehr �hnlich aufgebaut. Es werden zuerst die Werte (meist mithilfe von einer Bibliothek) eingelesen und anschlie�end in das Kommunikationsobjekt gespeichert, oder wie im Fall der L�fter, aus dem Objekt ausgelesen und �bernommen. Der Intervall indem Daten ausgelesen werden, kann au�erdem f�r alle Sensoren festgelegt werden.
					
					\lstset{style=VS2017}
					\begin{lstlisting}[language=C, caption={Einlesen der Spannungssensoren}]	
void VoltageSensor() {
  
  currentMillisVoltage = millis();
  if(currentMillisVoltage - previousMillisVoltage >= ReadIntervallVoltage) {
    previousMillisVoltage = currentMillisVoltage;

    for(int i = 0;i < NOSVoltage;i++) {                         
      float value = analogRead(VoltagePin[i]);          //Reads all the AnalogPins Values
			
      //Calculates the voltages from the sensorpins values
      LattePanda.Voltage.Data[i] = value*(LogicLevel/pow(2, ADCRes))*((R1[i] + R2[i])/R2[i]);    
    }
    
  }
}					
					\end{lstlisting}
					
					\lstset{style=VS2017}
					\begin{lstlisting}[language=C, caption={Steuern der L�fter}]	
void Fan_Control() {

  for(int i = 0;i < 4;i++) {
    
    FanSpeed[i] = LattePanda.Fan.Data[i];   //Read Data from LattePanda
		
    //Calculate DutyCycle from Speed Value 
    DutyCycle[i] = map(FanSpeed[i], 0, 100, 0, pow(2, Resolution[i]));    

    ledcWrite(PWMChannel[i], DutyCycle[i]);     //Write PWM Signal to Pins

  }

}					
					\end{lstlisting}
					Die Loop-Schleife von Kern 0 ruft dauerhaft folgende Funktionen auf:
					
					\lstset{style=VS2017}
					\begin{lstlisting}[language=C, caption={Steuern der L�fter}]	
void Task1loop() {
  
  CheckError();
  ReadTemp();
  ReadSonar();
  ReadIMU();
  Fan_Control();
  VoltageSensor();
  getSerialData();
	SendSerialData();

  delay(1);

}					
					\end{lstlisting}
					