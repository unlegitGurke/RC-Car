\fancyfoot[C]{Sandri}
\section{Programmierung des ESP32-Mikrocontrollers}
	Die Aufgaben des ESP32 bestehen aus dem Auslesen aller Sensordaten, dem Steuern der Beleuchtung und der Kommunikation mit dem Bordcomputer. Da der ESP32 2 Kerne besitzt wird der Code zwischen diesen aufgeteilt, das Einlesen aller Sensoren und die Kommunikation l�uft auf Kern 0, das Programm f�r die Beleuchtung auf Kern 1. Um den beiden Kernen getrennt Aufgaben zuordnen zu k�nnen, m�ssen im void Setup() 2 Tasks, welche an einen bestimmten Kern gebunden sind, erstellt werden.
	
	\lstset{style=VS2017}
	\begin{lstlisting}[language=C, caption={ESP32 2 Kerne Setup}]
void setup() {

	Serial.begin(115200);

	xTaskCreatePinnedToCore(
				Task1setup,   /* Task function. */
				"Task1",     /* name of task. */
				10000,       /* Stack size of task */
				NULL,        /* parameter of the task */
				1,           /* priority of the task */
				&Task1,      /* Task handle to keep track of created task */
				0);          /* pin task to core 0 */                  
	delay(500); 

	xTaskCreatePinnedToCore(
				Task2setup,   /* Task function. */
				"Task2",     /* name of task. */
				10000,       /* Stack size of task */
				NULL,        /* parameter of the task */
				1,           /* priority of the task */
				&Task2,      /* Task handle to keep track of created task */
				1);          /* pin task to core 1 */
	delay(500);

}
	\end{lstlisting}
	

	\subsection{Kern 0 / Sensoren und Kommunikation}
	Die Hauptaufgabe von Kern 0 ist es, mit dem Bordcomputer zu kommuniziern, alle Sensordaten auszulesen, und Komponenten, wie zum Beispiel L�fter, zu steuern.
	Die Seriellen Schnittstellen vom ESP32 und dem Bordcomputer sind mithilfe des Shields verbunden, um eine fl�ssige Kommunikation zu gew�hrleisten, ist ein Protokoll n�tig:
	
	\subsubsection{Aufbau von Paketen zwischen ESP32 und Bordcomputer}
		Der grunds�tzliche Aufbau eines Paktes zwischen ESP32 und Bordcomputer ist immer gleich, er besteht aus folgenden Teilen
			
			\begin{enumerate}
			
				\item Datentyp: LED, Temp, Volt, IMU1, IMU2, Sonar, FAN, Error //Alle m�glichen Daten, die zwischen ESP und Bordcomputer gesendet werden k�nnen. 
				
				\item Anzahl der Datenwerte
				
				\item Access, Read / Write / Both  //Sollen die Daten nur eingelesen werden, wird nur eine Antwort erwartet oder wird beides erwartet?
				
				\item Daten, je nach Anzahl der Datenwerte k�nnen hier beliebig viele Werte hintereinander geschrieben werden
				
				\item Error, wird ein Fehler gemeldet?
			
			\end{enumerate}
			
			Jedes Datenpaket beginnt mit einem Marker und endet mit einem Marker, in diesem Fall wurde als StartMarker "x" verwendet und als EndMarker "y". Die Daten sind in dazwischen in genannter Reihenfolge mit Beistrichen getrennt.
			Die Gr��e der verschiedenen Datenpaktet ist wiefolgt:
			
			\begin{table}[H]
				\centering
				\caption{Datenpakete}
				
				\begin{tabular}{|c|c|c|c|c|c|}
					
					\hline
					\multicolumn{6}{|c|}{\textbf{Datenpakete}}\\
					\hline
					\hline
					\textbf{Name} & \textbf{Datentyp} & \textbf{Anzahl der Variablen} & \textbf{Access} & \textbf{Daten} & \textbf{Error} \\
					\hline
					IMU1 & 1 & 6 & 1 & - & 0 \\
					\hline
					IMU2 & 2 & 3 & 1 & - & 0 \\
					\hline
					OctoSonar & 3 & 16 & 1 & - & 0 \\
					\hline
					Spannungssensoren & 4 & 5 & 1 & - & 0 \\
					\hline
					Temperatursensoren & 5 & 6 & 1 & - & 0 \\
					\hline
					L�fter & 6 & 4 & 2 & - & 0 \\
					\hline
					LEDs & 7 & 4 & 2 & - & 0 \\
					\hline
					IsOK & 0 & 0 & 2 & - & 0 \\
					\hline
					
				\end{tabular}
			\end{table}
			
			Access = 0 bedeutet, dass Daten nur vom ESP gelesen werden und es keine Antwort gibt.\\
			Access = 1 bedeutet, dass Daten an den Bordcomputer vom ESP geschickt werden sollen.\\
			Access = 2 bedeutet, dass Daten vom ESP eingelesen werden sollen und dass eine Antwort geschickt werden soll.\\
			\\
			Error = 0 bedeutet, das kein Fehler vorhanden ist.\\
			Error = 1 bedeutet, dass eine Warnung vorhanden ist.\\
			Error = 2 bedeutet, dass ein fataler Fehler vorhanden ist.\\
			
		\subsubsection{Einlesen der Pakete am ESP32}
			F�r die Kommunikation zwischen ESP32 und Bordcomputer ist eine eigene Bibliothek verantwortlich, so muss im Hauptprogramm nur eine Objekt "Computer" erstellt werden, um Daten einzulesen beziehungsweise zu empfangen.
			Die erste Funktion der Bilbiothek lie�t die Daten ein und speichert sie in einem Buffer:
			
			\lstset{style=VS2017}
			\begin{lstlisting}[language=C, caption={Einlesen eines Pakets}]
bool LattePandacomms::refresh(char startMarker, char endMarker) {
  
  _inProgress = false;
  
  while (Serial.available() > 0) {    //Checks for Serial Data
    
    byte x = Serial.read();   //Reads Serial Data
    
    if (x == startMarker && _inProgress == false) {    //IF the first Byte = startMarker, start to recieve Data
       
      _bytesRecvd = 0;      //Zero bytes have been recieved so far
      _inProgress = true;   //New Message has started
      
      for(int i = 0; i < sizeof(BufferIn); i++) {   //Clear BufferIn Array to make space for new message
        BufferIn[i] = '\0';
      }
      
    }
    
    if (_inProgress ) {   //Has the message started?
          
      if (_bytesRecvd<_maxMessage) {    //Does the current length exceed the max Message limit
        
        BufferIn[_bytesRecvd++] = x;  //Save current Byte to Buffer
        
      }
      
      if (x == endMarker) {     //End of message 
        
        _inProgress = false;    //Reset inprogress variable
        //allReceived = true;
        _nb = _bytesRecvd;
        return 1;
        
      }
      
    }

    
  }
  
  //Serial.println(BufferIn);   //DEBUG
  return 0;
  
}			
			\end{lstlisting}
			
			Diese Funktion wartet auf den StartMarker und speichert anschlie�end bis zum Erscheinen des EndMarkers jede Ziffer in ein Buffer-Array, welches danach weiterverarbeitet werden kann.
			
			\subsubsection{Dekodieren der Pakete am ESP32}
					Um das sich im Buffer befindliche Paket nun zu dekodieren sind zwei zus�tzliche Funktionen n�tig:\\
					Die Erste Funktion entfernt den Start-und Endmarker, da diese nicht dekodiert werden m�ssen.\\
					Mithilfe von der zweiten Funktion ist m�glich den ersten Wert aus einem Char-Array zu l�schen. So kann nun immer der erste Wert eingelesen werden und dann gel�scht werden. Dies wird so oft widerholt, bis die gew�nschte Anzahl an Variablen(durch Anzahl der Variablen im Paket gegeben) eingelesen ist.\\
					Diese eingelesen Variablen werden in Komponenten, die durch ein struct definiert sind, im public Teil der Class abgespeichert, so ist es m�glich vom Hauptcode auf diese zuzugreifen:
					
					\lstset{style=VS2017}
					\begin{lstlisting}[language=C, caption={Datenspeicherort und Struktur}]
struct Component {    //Struktur des Datenspeicherorts
  uint8_t Type;
  uint8_t nVal;
  uint8_t Access;
  uint8_t Data[maxnDataVar];
  uint8_t Error;
};

class LattePandacomms {
public:
  
  LattePandacomms();

  Component IsOk;
  Component IMU1;
  Component IMU2;
  Component Octosonar;
  Component Voltage;
  Component Temp;
  Component Fan;
  Component LED;
}						
					\end{lstlisting}
					
			\subsubsection{}		
	
	\subsection{Kern 1 / Beleuchtung}
	Das Programm auf Kern 1 ist daf�r zust�ndig die Fahrzeugbeleuchtung zu steuern. Vom Computer wird ein Effekt empfangen, anhand von welchem dann Blinker, Bremslicht oder R�cklicht eingeschalten werden sollen. Um den guten Ablauf von Animationen zu erlauben, ist es wichtig, dass keine delay(); Funktion im Code verwendet wird. Zum Ansteuern der LEDs wird die FastLED-Bibliothek verwendet. Diese erm�glicht das leichte Steuern von adressierbaren LED-Streifen. Der Code ist in einige Teile aufgeteilt:
	
		\subsubsection{Setup}
			Im Setup werden als Erstes der vordere und hintere LED-Streifen initialisiert. Die verwendeten Streifen sind WS2812B-Streifen und haben 45, beziehungsweise 34LEDs. Danach werden die LEDs ausgeschalten und es wird die Startup-Animation initialisiert. Wenn das Setup abgeschlossen ist, l�uft es in eine unendliche for-Schleife in welcher der Rest des Programmes ausgef�hrt wird.
			
			\lstset{style=VS2017}
			\begin{lstlisting}[language=C, caption={Steuern der L�fter}]	
void Task2setup( void * pvParameters ) {
  
	//Initlialize LEDStrips
  FastLED.addLeds<WS2812B, LED_PIN_BACK, GRB>(ledsback, NUM_LEDS_BACK);  
  FastLED.addLeds<WS2812B, LED_PIN_FRONT, GRB>(ledsfront, NUM_LEDS_FRONT);  
    
  pinMode(LED_PIN_BACK, OUTPUT);              
  pinMode(LED_PIN_FRONT, OUTPUT);

  fill_solid(ledsback, NUM_LEDS_BACK, CRGB::Black);    //Turn off all LEDs at startup
  fill_solid(ledsfront, NUM_LEDS_FRONT, CRGB::Black);
  FastLED.show();

  IsStartup = 1;    //Run Startup Animation

  for(;;){
    Task2loop();
  }
    
}			
			\end{lstlisting}
			
		\subsubsection{Loop}
			Der Loop pr�ft als Erstes ob die Startup-Animation l�uft, wenn ja spielt er sie bis zum Ende ab. Wenn diese fertig ist, wird eingelesen, welcher Effekt derzeit auf der Beleuchtung abgespielt werden soll. Wenn kein Effekt aufgerufen wurde, wird die \textit{idle();} Funktion aufgerufen, in dieser ist das Standlicht definiert. Wenn ein Effekt gew�nscht ist werden hier die entsprechenden Funktionen wie der Blinker, das Bremslicht, oder das R�cklicht aufgerufen. Diese Effekte k�nnen auch beliebig kombiniert werden.
		
		\subsubsection{Blinker}
			Die Funktion Blinker pr�ft als Erstes ob  der linke Blinker, der rechter Blinker oder beide laufen sollen. Danach startet er eine Ablaufsteuerung in der die Lauflichtanimation abgespielt wird. Anstatt die delay(); Funktion zu verwenden wird auch hier die millis(); Funktion verwendet um mehrere Effekte gleiche Effekt zu erm�glichen. Alle anderen Funktionen, wie R�cklicht und Bremslicht, funktionieren �hnlich wie die Blinker Funktion.
			
		\subsubsection{FadeToColor}
			Die FadeToColor Funktion kann wird innerhalb von anderen Funktionen aufgerufen um die Farbe einer oder mehreren LEDs auf eine andere Farbe innerhalb eines Intervalls zu �ndern. Daf�r werden zuerst f�r rot, gr�n und blau die Differenzen zwischen akuteller Farbe und Zielfarbe berechnet. Anhand der gr��ten Differenz, wird eine for-Schleife gestartet, in der die Variablen einander angen�hert werden.
