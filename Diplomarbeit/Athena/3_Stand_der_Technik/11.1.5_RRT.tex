\subsubsection{RRT (Rapidly-exploring Random Tree) Pathfinding-Algorithmus}
Der RRT-Algorithmus ist eine fortschrittliche Methode zur Berechnung von Pfaden in einem unbekannten oder komplexen Umfeld. Dieser Algorithmus wurde f�r autonome Systeme entwickelt, um effiziente und kollisionsfreie Routen zu planen, insbesondere in dynamischen Umgebungen wie Robotern oder autonomen Fahrzeugen.

\textbf{Funktionsweise des RRT-Algorithmus:}
Der RRT-Algorithmus arbeitet, indem er einen Baum von Pfaden im Umfeld des Startpunkts aufbaut und diesen Baum schrittweise erweitert, um den Zielpunkt zu erreichen. Dabei werden zuf�llige Punkte im Umfeld generiert und mit dem vorhandenen Baum verbunden. Diese zuf�lligen Punkte dienen dazu, das Umfeld zu erkunden und potenzielle Wege zum Ziel zu entdecken.

\textbf{Erweiterung des Baumes:}
Bei jedem Schritt des Algorithmus wird der n�chste Punkt in Richtung des Ziels erweitert. Dies geschieht durch das Hinzuf�gen eines neuen Knotens zum Baum, der durch eine Verbindung mit einem bestehenden Knoten erreicht wird. Die Auswahl des n�chsten Punktes erfolgt in der Regel basierend auf der N�he zum zuf�llig generierten Punkt und der Sicherstellung, dass die Verbindung zwischen den Punkten kollisionsfrei ist.

\textbf{Dynamische Umgebungen:}
Der RRT-Algorithmus ist besonders effektiv in dynamischen Umgebungen, da er in der Lage ist, sich an Ver�nderungen anzupassen und neue Hindernisse zu ber�cksichtigen. Durch kontinuierliches Erweitern des Baumes kann der Algorithmus neue Wege finden, um Hindernisse zu umgehen oder alternative Routen zu planen, wenn sich die Umgebung �ndert.

\textbf{Optimierung und Feinabstimmung:}
Nachdem der Baum aufgebaut wurde und der Zielpunkt erreicht wurde, k�nnen verschiedene Optimierungstechniken angewendet werden, um den gefundenen Pfad zu verbessern. Dazu geh�ren beispielsweise die Gl�ttung des Pfades, um unn�tige Wendungen zu reduzieren, oder die Anpassung des Pfades an spezifische Anforderungen oder Einschr�nkungen.

\textbf{Anwendungen des RRT-Algorithmus:}
Der RRT-Algorithmus wird in einer Vielzahl von Anwendungen eingesetzt, darunter autonomes Fahren, Robotik, Videospielentwicklung und Luft- und Raumfahrt. Durch seine F�higkeit, schnell und effizient kollisionsfreie Pfade zu berechnen, bietet der RRT-Algorithmus eine leistungsstarke L�sung f�r komplexe Navigationsprobleme in dynamischen Umgebungen.
