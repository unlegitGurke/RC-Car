\subsection{Radar Sensoren}
Radar (Radio Detection and Ranging) Sensoren bestehen aus einer Sendeantenne und einer Empfangsantenne. Die Sendeantenne sendet hochfrequente elektromagnetische Wellen aus, die von Objekten reflektiert werden und von der Empfangsantenne erfasst werden. Die Zeitmessung zwischen Aussenden und Empfangen des Signals erm�glicht die Berechnung der Entfernung zum Objekt. 

Radar bietet eine effektive M�glichkeit, Entfernungen zu messen und Bewegungen zu verfolgen. Es wird h�ufig in der Luft- und Seefahrt, der Wetterbeobachtung, der Verkehrs�berwachung und der milit�rischen Anwendung eingesetzt. Im Vergleich zu LiDAR und Ultraschallsensoren bietet Radar eine gr��ere Reichweite und ist weniger anf�llig f�r Umwelteinfl�sse wie Nebel oder Regen.
