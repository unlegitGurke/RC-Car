\fancyfoot[C]{G�rel}
\subsection{Einplatinencomputer}

Ein Einplatinencomputer ist eine kompakte Computerplatine, die alle wichtigen Komponenten eines Computers enth�lt, einschlie�lich Prozessor, Speicher, E/A-Ports und vieles mehr. Diese Computer sind oft kosteng�nstig und vielseitig einsetzbar, was sie f�r eine Vielzahl von Anwendungen attraktiv macht.

\subsubsection{Raspberry Pi}

Der Raspberry Pi ist eine Serie von Einplatinencomputern, die alle mit einer ARM Cortex CPU betrieben werden. Die Firma Raspberry Pi ist vor allem in der Hobbyzone f�r ihre kosteng�nstigen und zuverl�ssigen Systeme bekannt. Diese kleinen, aber leistungsstarken Computer bieten eine Vielzahl von Anwendungsm�glichkeiten, von der Heimautomatisierung �ber die Robotik bis hin zur Entwicklung von Prototypen f�r IoT-Ger�te. Sie sind f�r ihre einfache Handhabung und ihre breite Unterst�tzung in der Entwicklergemeinschaft beliebt.

\subsubsection{LattePanda}

Der LattePanda ist eine Serie von Einplatinencomputern, die die x86-Architektur mit Intel-CPUs verwenden. Der im Projekt verwendete LattePanda 3 Delta verf�gt �ber zwei M.2-Steckpl�tze (PCIe und SATA), von denen einer mit der Google Coral TPU belegt worden ist. Diese Kombination erm�glicht die Rechenleistung f�r s�mtliche ben�tigten KI-Anwendungen. Der Einsatz von x86-Architektur und Intel-CPUs bietet eine breite Kompatibilit�t mit vorhandener Software und erleichtert die Integration von externen Ger�ten und Erweiterungen. Der LattePanda Delta 3 bietet somit eine leistungsstarke Plattform f�r KI-Anwendungen und andere rechenintensive Projekte.

\subsubsection{Nvidia Jetson}

Der Nvidia Jetson ist ein System-on-a-Module (SoM) mit einer NVIDIA Recheneinheit, welche speziell auf KI-Anwendungen ausgerichtet ist. Aufgrund seiner ausgezeichneten Leistungsf�higkeit und seiner speziellen Ausrichtung auf KI-Berechnungen w�re der Jetson f�r das Projekt ideal geeignet. Allerdings sind die Kosten f�r den Jetson vergleichsweise hoch, was ihn als Option ausschlie�t. Der Nvidia Jetson bietet eine solide Leistung f�r KI-Anwendungen, ist jedoch m�glicherweise nicht die kosteng�nstigste Option f�r das Projekt.

Im Vergleich dazu bietet der LattePanda Delta 3 eine kosteng�nstigere Alternative mit den erforderlichen Funktionen f�r das Projekt. Durch die Verwendung von Intel-CPUs und der x86-Architektur erm�glicht der LattePanda Delta 3 dennoch eine solide Leistung f�r KI-Anwendungen. Die Verf�gbarkeit von zwei M.2-Steckpl�tzen, von denen einer mit der Google Coral TPU belegt ist, bietet die erforderliche Rechenleistung f�r die geplanten KI-Anwendungen. Obwohl der LattePanda Delta 3 nicht speziell f�r KI-Anwendungen entwickelt wurde, bietet er dennoch eine kosteng�nstige L�sung mit ausreichender Rechenleistung und Erweiterungsm�glichkeiten f�r das Projekt.
