\subsection{Filamente des FDM-Drucks}
	Beim 3D-Druck spielt das verwendete Filament eine sehr gro�e Rolle. Durch die korrekte Auswahl dessen k�nnen sehr viele verschiedene mechanische Eigenschaften erreicht werden.
	\\
	\\
	Zum FDM-Druck werden Thermoplaste in Form einer Schnur auf eine Rolle aufgewickelt verwendet. Dieses Filament wird meist im Querschnitt 1,75mm und 2.85mm angeboten, in den letzten Jahren hat sich jedoch der Querschnit 1,75mm weitl�ufig durchgesetzt. Filament-Spulen sind meist in 1-Kilo-Inkrementen erh�ltlich.
	
	\subsubsection{Polylactid (PLA)}
		Polylactid, meist unter seiner Abk�rzung PLA bekannt, ist der am weitesten verbreitete Kunststoff im FDM-Druck. Er ist vorallem durch seine leichte Verarbeitung und geringen Preis weit verbreitet worden. Hergestellt wird PLA aus Pflanzenst�rke. PLA ist grunds�tzlich transparent, steif, �L-, Fett- und Alkoholbest�ndig und zugellasen f�r Lebensmittel. Jedoch ist es spr�de, nicht UV-best�ndig und leicht thermisch verformbar.\cite{kunststoffe.de2024}
		
	\subsubsection{Polyethylenterephthalat (PET/PETG/PETT)}
		Polyethylenterephthalat ist ein sehr weit verbreiteter Kunststoff, er ist aus allem von Recycling-Flaschen bis zu Textilfa�ern bekannt. Beim 3D-Druck wird reines PET eher selten verwendet, es wird die abge�nderte Variante PETG verwendet. Diese ist mit Glycol modifiziert und sorgt daf�r, dass das Material besser f�r den 3D-Druck geeignet wird. Es wird oft als bessere Alternative zu PLA betrachtet, da es haltbarer und flexibler als PLA ist aber immernoch sehr einfach zu drucken ist. Polyethylen-CoTrimethylen-Terephthalat (PETT) wird auch immer h�ufiger vorallem aufgrund von seiner Transparenz verwendet.\cite{all3dp.com2023}
	
	\subsubsection{Acrylnitril-Butadien-Styrol-Copolymer(ABS)}	
		Acrylnitril-Butadien-Styrol-Copolymer, auch unter ABS bekannt, ist der g�ngigste Werkstoff wenn 3D-Druck-Teile etwas widerstandsf�higer sein m�ssen. ABS hat hohe Festigkeit, eine h�here Temperaturbest�ndigkeit wie PLA und PETG und ist nicht spr�de. Jedoch ist es auch erheblich schwieriger zu drucken, da es dazu neigt sich beim Abk�hlen zu verformen, deshalb wird beim Drucken meist eine recht hohe Druckbetttemperatur von 80-110�C verwendet. Auch ein geschlossener Druckraum ist aus diesem Grund von Vorteil. Beim Druck enstehen auch D�mpfe die zu Kopfschmerzen oder �belkeit f�hren k�nnen, deswegen ist eine gute Entl�ftung beziehungsweise Filttrierung der Luft zu empfehlen.\cite{all3dp.com2023}
	
	\subsubsection{Thermoplastische Elastomere(TPE/TPU)}
		Thermoplastische Elastomere sind Kunststoffe mit Gummiartigen Eigenschaften, die sehr flexibel und sehr haltbar sind. TPU ist eines Sort von TPE, die etwas steifer und besser f�r den 3D-Druck geeignet ist. Sie k�nnen St��en und Ersch�tterungen wesentlich besser als PLA oder ABS standhalten.\cite{all3dp.com2023} TPU ist grunds�tzlich einfach zu drucken, St�tzsturkturen sollten jedoch weitgehend vermieden werden, da die klebrige und flexible Natur des Materials diese sehr schwierig zu entfernen macht. TPU ist auch wie viele andere Thermoplate hygroskopisch, wenn es vor dem Druck nicht ausreichend trocken ist, kann dies zu starkem Stringing, zur Fadenbildung f�hren.