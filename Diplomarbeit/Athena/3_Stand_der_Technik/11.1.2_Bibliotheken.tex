\subsection{Code-Bibliotheken}
Bei Projekten jeglicher Gr��e stellen Softwarebibliotheken eine entscheidende Ressource dar. Diese Bibliotheken bieten vorgefertigte Codefragmente, Funktionen und Tools, die Entwickler verwenden k�nnen, um bestimmte Aufgaben effizienter zu erledigen. Bibliotheken k�nnen eine Vielzahl von Zwecken erf�llen und dabei helfen, komplexe Probleme zu l�sen. Hier sind einige wichtige Aspekte zu beachten:

\textbf{Beschleunigung der Entwicklung: }
Die Verwendung von Softwarebibliotheken erm�glicht es Entwicklern, Zeit zu sparen, indem sie bereits existierende L�sungen nutzen, anstatt alles von Grund auf neu zu erstellen. Dies beschleunigt den Entwicklungsprozess erheblich und erlaubt es den Entwicklern, sich auf die spezifischen Anforderungen ihres Projekts zu konzentrieren, anstatt sich mit allgemeinen Implementierungsdetails zu befassen.

\textbf{Qualit�t und Zuverl�ssigkeit: }
Viele Softwarebibliotheken werden von einer gro�en Entwicklergemeinschaft unterst�tzt und haben sich im Laufe der Zeit bew�hrt. Sie sind oft gr�ndlich getestet und bieten eine hohe Qualit�t und Zuverl�ssigkeit. Dies erm�glicht es den Entwicklern, auf bew�hrte L�sungen zur�ckzugreifen und potenzielle Fehler oder Probleme zu minimieren.

\textbf{Anpassbarkeit und Erweiterbarkeit: }
Obwohl Softwarebibliotheken vorgefertigte L�sungen bieten, sind sie oft auch anpassbar und erweiterbar. Entwickler k�nnen die Bibliotheken entsprechend ihren spezifischen Anforderungen konfigurieren oder anpassen und bei Bedarf neue Funktionen hinzuf�gen. Dies erm�glicht es den Entwicklern, die Bibliotheken optimal an ihre individuellen Bed�rfnisse anzupassen.

\textbf{Entwicklung von eigenen Bibliotheken: }
Entwickler erstellen auch eigene Bibliotheken, um ihren Code aufzuteilen und die Wiederverwendbarkeit zu erh�hen. Durch das Auslagern von h�ufig verwendeten Funktionen oder Modulen in separate Bibliotheken k�nnen Entwickler ihren Code sauberer strukturieren und das Debugging vereinfachen. Dies kann Rechenleistung und Arbeitsspeicher einsparen. Au�erdem erm�glicht das Aufteilen des Codes die Zusammenarbeit an einem Projekt, da die Arbeit an einer Datei nicht sonderlich effektiv ist.

\textbf{Lizenzierung und rechtliche Aspekte: }
Bei der Verwendung von Softwarebibliotheken ist es wichtig, die Lizenzbedingungen zu beachten und sicherzustellen, dass die Nutzung im Rahmen der Lizenzbestimmungen erfolgt. Einige Bibliotheken erfordern m�glicherweise eine bestimmte Art der Lizenzierung oder setzen bestimmte Einschr�nkungen hinsichtlich der kommerziellen Nutzung. Die meisten der Open-Source Bibliotheken m�ssen bei Arbeiten zitiert werden. Die finalisierten Projekte d�rfen bei den g�ngigsten Lizenzvereinbarungen nicht mit einer Lizenz weiterver�ffentlicht werden, welche die Rechte weiter einschr�nkt. Ein Beispiel dazu ist, eine Open-Source Bibliothek hat die Creative Commons Share-Alike 4.0 Lizenz. Man darf diese Bibliothek benutzen und modifizieren. Jegliche modifizierten Versionen oder Projekte, die damit geschrieben wurden, m�ssen den Code mit der gleichen Lizenz ver�ffentlichen.
