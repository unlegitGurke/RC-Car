\subsection{Sicherheitsvorkehrungen}
	Die Strahlung von Lasern kann sehr schnell f�r Auge und Haut gef�hrlich werden, deswegen gibt es einige wichtige Sicherheitsvorkehrungen, die eingehalten werden m�ssen.
	\begin{itemize}
		\item Laserstrahl nie auf andere Personen richten
		\item Laserstrahl nie auf direkt reflektierende Oberfl�chen richten
		\item Nicht in den direkten oder reflektierten Strahl blicken
		\item Keine optischen Instrumente(z.B. Lupe oder Fernglas) zur Beobachtung des Laserstrahl verwenden
		\item Niemals die Laserquelle manipulieren.
	\end{itemize}
	Laserger�te werden au�erdem vom Hersteller gem�� ihrem Gef�hrdungspotenzial in verschiedene Klassen eingeteilt. Ma�geblich daf�r ist die DIN-Norm EN 60825-1. Die Klassen sind so eingeteilt, dass mit zunehmender Gefahr die Klassen h�her wird. Diese Klassen reichen von 1 bis 4 und sind in 1, 1M, 1C, 2, 2M, 3A, 3R, 3B und 4 aufgeteilt.\cite{Strahlenschutz2024}
	
	\begin{figure}[H]
		\centering
		\includegraphics[scale=0.5]{./3_Stand_der_Technik/Abbildungen/Laserklassen_1}
		\caption{Laser Sicherheitsklassen\cite{eval.at2024a}}
	\end{figure}