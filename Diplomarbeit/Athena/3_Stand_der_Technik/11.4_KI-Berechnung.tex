\fancyfoot[C]{G�rel}
\subsection{KI-Berechnungen}

K�nstliche Intelligenz (KI) basiert auf komplexen mathematischen Modellen und Algorithmen, die es Computern erm�glichen, menschen�hnliche Entscheidungen zu treffen oder Muster in Daten zu erkennen. Viele dieser Algorithmen verwenden Matrixberechnungen als Grundlage f�r ihre Funktionsweise.

\paragraph{Warum sind KI-Berechnungen Matrixberechnungen?}

KI-Algorithmen, insbesondere neuronale Netzwerke, verwenden Matrixoperationen, um Daten zu verarbeiten und Muster zu erkennen. Ein neuronales Netzwerk besteht aus mehreren Schichten von Neuronen, die miteinander verbunden sind. Jedes Neuron nimmt Eingaben von anderen Neuronen entgegen, multipliziert sie mit Gewichtungen und f�hrt eine Aktivierungsfunktion aus, um ein Ergebnis zu erzeugen. Diese Operationen k�nnen effizient in Form von Matrixmultiplikationen durchgef�hrt werden.

In einem neuronalen Netzwerk werden die Gewichtungen zwischen den Neuronen als Gewichtsmatrizen dargestellt. Die Eingaben werden ebenfalls in Form von Matrizen dargestellt. Durch die Anwendung von Matrixoperationen k�nnen komplexe Berechnungen parallelisiert und effizient auf Grafikprozessoren (GPU) oder speziell entwickelten Tensor Processing Units (TPU) durchgef�hrt werden. Durch das Ver�ndern dieser Gewichte wird eine KI �trainiert�, wobei in einer �Generation� mehrere Matrizen miteinander verglichen werden, und die beste Variante als Basis der n�chsten Generation ausgew�hlt wird.

\paragraph{Warum sind TPUs f�r KI-Berechnungen besser geeignet?}

Tensor Processing Units (TPUs) sind speziell f�r die Beschleunigung von KI-Berechnungen optimierte Hardwareeinheiten. Im Gegensatz zu herk�mmlichen CPUs oder GPUs sind TPUs darauf ausgelegt, Matrixoperationen effizient durchzuf�hren, was zu einer erheblichen Beschleunigung von KI-Anwendungen f�hrt.

TPUs sind in der Lage, gro�e Datenmengen schnell zu verarbeiten und komplexe neuronale Netzwerke in Echtzeit zu trainieren. Ihre Architektur ist darauf ausgerichtet, die spezifischen Anforderungen von KI-Anwendungen zu erf�llen.

\paragraph{Aufbau neuronaler Netze als Matrix}

Neuronale Netzwerke werden oft als Matrizen von Gewichtungen und Eingaben dargestellt. Jedes Neuron in einem neuronalen Netzwerk kann als Knoten in einer Schicht betrachtet werden, wobei die Verbindungen zwischen den Neuronen als Gewichtungsmatrizen repr�sentiert werden. Durch die Kombination von Matrixmultiplikationen und Aktivierungsfunktionen k�nnen neuronale Netzwerke komplexe Berechnungen durchf�hren und Muster in Daten erkennen.
