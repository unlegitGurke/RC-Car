\subsection{Gleichstrommaschine}
	Die Gleichstrommascheine besteht im Stator entweder aus Permanetmagneten, oder aus permanent-erregten Erregerwicklung. Der Rotor besteht aus Wicklungen, welche durch einen Kommutator (auch Stromwender genannt) und B�rsten mithilfe von Gleichspannung versorgt werden. Die Drehzahl der Maschine kann hierbei ganz einfach durch Ver�ndern der Versorgungsspannung linear geregelt werden. 
	
	\begin{figure}[H]
			\centering
			\includegraphics[scale=0.8]{./3_Stand_der_Technik/Abbildungen/Gleichstrommaschine_1}
			\caption{Aufbau Gleichstrommaschine\cite{Pischtschan2024}}
	\end{figure}
	
	Vorteil des Gleichstrommotors ist seine einfach Drehzahlregelung durch Ver�nderung der Spannung, Nachteil der hohe Wartungsaufwand und Verschlei� der Maschine. Er wird aufgrund von sinkenden Preisen bei Frequenzumrichtern immer weniger verwendet.