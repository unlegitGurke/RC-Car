\fancyfoot[C]{G�rel}
\subsection{Hardwarebeschleunigung durch Recheneinheiten}
Hardwarebeschleunigung spielt eine entscheidende Rolle 
bei der Leistung von KI-Anwendungen, \\
insbesondere bei der Verarbeitung gro�er Datenmengen und komplexer Modelle.
Drei Haupttypen von Recheneinheiten werden h�ufig in KI-Anwendungen verwendet: TPUs, GPUs und CPUs.

\paragraph{Coral TPU (Tensor Processing Unit): }
Die Coral TPU ist eine speziell entwickelte Hardwareeinheit von Google, die f�r die Beschleunigung von KI-Berechnungen optimiert ist. 
Sie ist darauf ausgerichtet, komplexe Berechnungen f�r neuronale Netzwerke schnell und effizient durchzuf�hren. 
Durch ihre Architektur k�nnen Coral TPUs Matrixoperationen in Echtzeit verarbeiten und sind besonders gut f�r inferenzbasierte Anwendungen wie Bilderkennung und Sprachverarbeitung geeignet. 
Coral TPUs sind f�r Edge-Computing-Szenarien konzipiert und erm�glichen die Implementierung von KI-Modellen direkt auf Ger�ten wie Smartphones, IoT-Ger�ten und Edge-Servern. 
Diese Modelle haben einen geringen Stromverbrauch bei moderater Leistung.

\paragraph{GPU (Graphics Processing Unit): }
Grafikprozessoren (GPUs) sind aufgrund ihrer hohen Parallelverarbeitungsf�higkeit und 
ihrer gro�en Anzahl von Rechenkernen ebenfalls beliebte Optionen f�r die Hardwarebeschleunigung von KI-Anwendungen. 
GPUs k�nnen gro�e Datenmengen effizient verarbeiten und sind besonders gut f�r das Training komplexer neuronaler Netzwerke geeignet. 
Sie bieten eine hohe Leistungsf�higkeit bei rechenintensiven Aufgaben wie dem Training von Deep-Learning-Modellen f�r Bilderkennung, Sprachverarbeitung und autonome Fahrzeuge. 
GPUs haben einen hohen Stromverbrauch bei hoher Leistung.

\paragraph{CPU (Central Processing Unit): }
Obwohl CPUs nicht so spezialisiert sind wie TPUs und GPUs, spielen sie dennoch eine wichtige Rolle bei der Ausf�hrung von KI-Anwendungen. CPUs sind vielseitig einsetzbar und k�nnen eine Vielzahl von Aufgaben bew�ltigen, einschlie�lich der Ausf�hrung von KI-Workloads. Sie eignen sich gut f�r inferenzbasierte Anwendungen mit niedrigerer Latenz und k�nnen auch f�r das Training kleinerer Modelle oder f�r weniger rechenintensive Aufgaben verwendet werden.

\paragraph{Auswahl der richtigen Recheneinheit: }
Die Auswahl der richtigen Recheneinheit h�ngt von den spezifischen Anforderungen der KI-Anwendung ab. F�r Edge-Computing-Szenarien mit begrenzter Energie- und Rechenleistung sind Coral TPUs aufgrund ihrer Effizienz und Leistungsf�higkeit eine gute Wahl. GPUs eignen sich hervorragend f�r das Training gro�er Modelle und komplexe Berechnungen auf Servern oder Workstations. CPUs bieten eine vielseitige L�sung f�r eine breite Palette von Anwendungen und k�nnen sowohl f�r inferenzbasierte als auch f�r Trainingsaufgaben eingesetzt werden.
