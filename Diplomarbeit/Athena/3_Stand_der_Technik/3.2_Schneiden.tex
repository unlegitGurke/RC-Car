\subsection{Schneiden verschiedener Werkstoffe}
	Laserschneidmaschinen sind meistens CNC-Maschinen. Um ein gutes Ergebnis zu erhalten, gilt es die Maschine dem Material entsprechend gut einzustellen. Die wichtigsten Faktoren sind dabei die Laserleistung und die Schnittgeschwindigkeit. Die hochwertigsten Ergebnisse werden meistens mit wenig Leistung und wenig Geschwindigkeit erreicht, sind bei gro�en Werkst�cken jedoch oft unvorteilhaft. Es gilt also ein Gleichgewicht zwischen Geschwindigkeit und Qualit�t der Schnittkante zu erreichen. 
	Bei gr��eren Materialdicken wird meistens auf die Schnittkante ein Gas mit hohem Druck zugef�hrt, um geschmolzenes Material zu entfernen und die Schnittkante daran zu hindern sofort zu oxidieren. Daf�r werden meistens entweder Druckluft, O2 oder N2 verwendet. 
	Die Reihenfolge der Schnittkonturen sollte auch beachtet werden. Im Normalfall werden zuerst alle innenliegenden Konturen und dann die au�enliegende Kontur geschnitten.