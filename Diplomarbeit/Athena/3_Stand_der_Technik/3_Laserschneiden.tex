\fancyfoot[C]{Sandri}
\section{Laserschneiden}
	Beim Laserschneiden k�nnen meist flache Werkstoffe verschiedenster Dicke mithilfe von einem konzentrierten Laserstrahl ohne Ber�hrung geschnitten werden. Der Laserstrahl erhitzt dabei das Material an einem Punkt so stark, dass es sofort schmilzt oder verdampft. Mithilfe von einem Schneidgas kann das nun geschmolzene Material aus der Schnittfuge geblasen werden. Mit dieser Technik k�nnen eine Vielzahl an Materialien, wie zum Beispiel Holz, Acryl, Schaumstoff oder auch diverse Metalle wie Stahl, Edelstahl und Aluminium. \cite{Trumpf2024}
	
	\subsection{Arten von Lasern}
	Test
	
	\subsection{Schneiden verschiedener Werkstoffe}
	Test