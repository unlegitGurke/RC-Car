\subsection{Energiespeicher}
	Das Ziel des Energiespeichers ist es m�glichst viel Energie in einem m�glichst kleinem Volumen zu speichern und diese effizient aufzunehmen und abzugeben. Der Aufbau dieser Batterien besteht meistens aus zwei unterschiedlichen Elektroden und aus einem dazwischen befindlichen Elektrolyt. Die Spannung einer solchen Zelle h�ngt von  der Elektronegativit�t der verschiedenen Elektroden ab. Die wichtigsten Eckdaten eines Akkus sind seine Spannung in V, seine Kapazit�t in Wh und die Endlade- und Laderate in C($\frac{Entladestrom}{Kapazit\ddot{a}t}$). Die meisten Akkus bestehen aus mehreren Zellen die in Serie oder parallel geschalten sind, um entweder Kapazit�t oder Spannung zu erh�hen. Die wichtigsten Akku-Typen sind folgende:
	
		\subsubsection{Alkali-Mangan-Zelle(AlMn)}
			Alkali-Mangan Batterien sind das am h�ufigsten verwendete Batteriesystem, sie �berzeugen vorallem mit ihrer hohen Verf�gbarkeit und g�nstigen Herstellung. Sie bieten auch eine recht hohe Energiedichte. Sie haben auch eine lange Haltbarkeit und ein niedrige Selbstentladung.
		
		\subsubsection{Silberoxid-Zelle(AgO)}
		
		\subsubsection{Lithium-Mangandioxid-Zelle(LiMnO2)}
		
		\subsubsection{Nickel-Cadmium-Zelle(NiCd)}
		
		\subsubsection{Nickel-Metallhybrid-Zelle(NiMH)}
		
		\subsubsection{Lithium-Ion-Zelle{Li-Ion}}
		
		\subsubsection{Lithium-Polymer-Zelle}