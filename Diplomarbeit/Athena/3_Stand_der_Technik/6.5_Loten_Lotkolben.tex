\fancyfoot[C]{G�rel}
\subsection{L�ten mit L�tkolben}

Beim traditionellen L�ten mit einem L�tkolben k�nnen erfolgreich Komponenten bis zu der Gr��e 0603 (0,06 x 0,03 Zoll) verl�tet werden. Allerdings gestaltet sich das Verl�ten von SMD ICs als schwierig. Die Hitze des L�tkolbens f�hrt teilweise zu Besch�digungen der ICs und zu visuellen sowie qualitativen Unregelm��igkeiten an den L�tstellen, was die Qualit�t und Zuverl�ssigkeit der Verbindungen beeintr�chtigt. Diese Probleme verdeutlichten die Notwendigkeit einer pr�ziseren und schonenderen L�ttechnik, insbesondere f�r empfindliche Bauteile wie SMD ICs.

Es gibt grunds�tzlich zwei Unterschiede beim L�tzinn: bleihaltig und bleifrei. Bleihaltiges L�tzinn hat eine niedrigere Schmelztemperatur von etwa 190�C, was eine einfachere Handhabung erm�glicht und gute Flusseigenschaften bietet. Jedoch birgt bleihaltiges L�tzinn Gesundheitsrisiken aufgrund der Freisetzung von Bleid�mpfen w�hrend des L�tprozesses.

Im Gegensatz dazu erfordert bleifreies L�tzinn eine h�here Schmelztemperatur von etwa 220�C. Diese erh�hte Temperatur kann dazu f�hren, dass empfindliche Komponenten w�hrend des L�tprozesses leichter besch�digt werden. Dennoch bietet bleifreies L�tzinn eine sicherere Alternative f�r die Gesundheit von Mensch und Umwelt, da es keine Toxine enth�lt.
