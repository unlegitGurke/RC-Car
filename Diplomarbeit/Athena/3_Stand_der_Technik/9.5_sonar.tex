\fancyfoot[C]{G�rel}
\subsection{Ultraschallsensoren}
Ultraschallsensoren bestehen aus einem Schallwandler und einer Empfangseinheit. Der Schallwandler sendet Ultraschallwellen aus, die an Objekten reflektiert werden und von der Empfangseinheit erfasst werden. 
Tonwellen ab 20 kHz gelten als Ultraschall und k�nnen demnach nicht vom Menschen geh�rt werden, die Sensoren operieren unter diesem Bereich \cite{Frantzen2024}. 
Die Zeitmessung zwischen Aussenden und Empfangen des Schallsignals erm�glicht die Berechnung der Entfernung zum Objekt. 

Ultraschallsensoren bieten eine kosteng�nstige L�sung f�r die Abstandsmessung und werden h�ufig als Fahrzeugparkassistenz oder in der Robotik verwendet. Der generelle Anwendungsbereich ist jener, bei dem eine grobe Ann�herung der Entfernung ausreicht.
