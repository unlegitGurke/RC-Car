\fancyfoot[C]{Sandri, Kaltenleitner, G�rel, Petrovic}
\chapter{Ergebnisse und Ausblick}

Das Projekt hat bedeutende Fortschritte gemacht, wobei verschiedene Teams erfolgreich an ihren jeweiligen Aufgaben gearbeitet haben. Die Teamleitung und Entwicklung des elektronischen Aufbaus, einschlie�lich des Kabelbaums und des Antriebssystems, wurden erfolgreich abgeschlossen. Ebenso wurde die Softwareentwicklung f�r den Mikrocontroller sowie der 3D-Druck der Bauteile erfolgreich durchgef�hrt.

Die Finanzierung des Projekts wurde abgeschlossen, und die Benutzeroberfl�chen sowie das Webdesign wurden erstellt und m�ssen nun integriert werden. Der mechanische Aufbau, das Design und der Aufbau des Antriebsstrangs, der Aufh�ngung, der Lenkung und der Karosserie wurden erfolgreich fertiggestellt, ebenso wie die Fertigung und der Zusammenbau der mechanischen Bauteile.

Bez�glich der Herausforderungen gab es Probleme mit der Spannungsversorgung aufgrund qualitativ minderwertiger Komponenten, was wiederholt zu Defekten an Sensoren f�hrte. Lieferzeiten und Bestelldauern einiger Komponenten stellten ebenfalls Hindernisse dar. Ein Kurzschluss auf der Platine war ebenfalls ein Problem, das angegangen werden musste.

F�r die Zukunft steht die Fertigstellung der Integration von Finanzierung, Benutzeroberfl�chen und Webdesign sowie die Entwicklung des ROS-Systems und die Umsetzung von Pathfinding, Personenerkennung und Steuerung des Fahrzeugs an. Mit diesen Schritten werden wir das Projekt abschlie�en und die gesteckten Ziele erreichen.
