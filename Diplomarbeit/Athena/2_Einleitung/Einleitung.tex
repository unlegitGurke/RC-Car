\chapter{Einleitung}
	
\section{Motivation}
	
Der Grundstein f�r Athena wurde in der zweiten Klasse gelegt. Es sollte ein CAD-Projekt im Fach CPE erstellt werden. Aufgrund unserer gro�er Automobilaffinit�t entschieden man sich f�r ein ferngesteuertes Auto. Da es in der zweiten Klasse jedoch nicht m�glich war, ein so komplexes Projekt zu fertigen, beschlossen wir, es weiterzuentwickeln.\\
\\
Um mehr in den Bereich Elektrotechnik einzutauchen, wurde ein Konzept f�r ein selbstfahrendes Auto entwickelt. Dieser Bereich ist derzeit besonders interessant, da viele gro�e Automobilhersteller bereits am autonomen Fahren arbeiten.\\
\\
In der vierten Klasse bot sich die M�glichkeit, dieses Projekt umzusetzen. Da dies aufgrund von Budgetknappheit und langen Lieferzeiten nicht m�glich war, wurde das Projekt in der f�nften Klasse als Diplomarbeit weitergef�hrt. Das Ziel war es, dieses Projekt nach jahrelanger Arbeit erfolgreich abzuschlie�en.

\section{Zielsetzung}

Die Zielsetzung f�r Athena besteht darin, ein innovatives und vollst�ndig autonomes Modellauto zu entwickeln, welches die Schulumgebung pr�zise kartieren und intelligent navigieren kann. Durch die Integration von K�nstlicher Intelligenz soll ein Fahrzeug geschaffen werden, das nicht nur in der Lage ist, Hindernisse zu erkennen und zu umgehen, sondern auch effiziente Routen innerhalb des Schulumfeldes planen kann.


\section{Aufbau}

Die Diplomarbeit ist in zwei �bergeordnete Bereiche unterteilt, wobei der praktische Teil drei spezifische Bereiche umfasst:\\

Im theoretischen Hauptteil erl�utern alle Autoren die grundlegende Theorie, die ihre Arbeit informiert. Dabei pr�sentiert jeder Autor eine umfassende Darstellung der theoretischen Grundlagen. Jeder Autor bietet eine ausf�hrliche Darstellung der theoretischen Grundlagen, die als Basis f�r ihre Arbeit dienen.\\

Die Unterbereiche umfassen:

\begin{itemize}
\item Elektronik
\item Mechanik
\item Software
\end{itemize}

Innerhalb der Arbeit werden die Elektronik, Mechanik und Software des Fahrzeugs behandelt. Dabei wird ein spezifischer Fokus auf elektronische Komponenten wie Motoren sowie auf mechanische Aspekte wie die Aufh�ngung gelegt. Die Softwareanalyse konzentriert sich auf die Programmierung des Fahrzeugs.
