\fancyfoot[C]{G�rel}
\section{Nutzung von Nodes in ROS}

Die Integration von Programmen in ROS erfordert die Verwendung von Konfigurations- und Launch-Dateien. Diese Dateien legen die Rahmenbedingungen f�r den Betrieb des ROS-Systems fest, konfigurieren die verschiedenen Komponenten und starten die erforderlichen ROS-Knoten. Diese Dateien sind �blicherweise im \texttt{catkin\_ws/src/(node)}-Verzeichnis des ROS-Arbeitsbereichs zu finden.

\paragraph{Konfigurationsdateien}

Um zum Beispiel den Google Cartographer zu verwenden, m�ssen Konfigurationsdateien erstellt werden, die die Parameter f�r den SLAM-Prozess festlegen. Die \texttt{athena\_config.lua}-Datei im Verzeichnis \texttt{cartographer\_ros/configuration\_files} definiert grundlegende Einstellungen.

\paragraph{ROS-Launch-Datei vorbereiten}

Die ROS-Launch-Datei dient dazu, die erforderlichen ROS-Knoten zu starten und die Konfiguration f�r den Google Cartographer bereitzustellen. In der Launch-Datei werden Parameter wie der URDF des Roboters, die Konfiguration der Sensoren und die Pfadangaben zu den Konfigurationsdateien festgelegt. Anschlie�end werden die ROS-Knoten gestartet, darunter der \texttt{robot\_state\_publisher} f�r das URDF, der \texttt{ydlidar\_ros\_driver\_node} f�r den Lidar-Sensor, statische Transformationsknoten und schlie�lich der \texttt{cartographer\_node} f�r den Google Cartographer selbst.

\paragraph{Konfiguration der \texttt{athena\_config.lua}-Datei}

Die \texttt{athena\_config.lua}-Datei enth�lt grundlegende Einstellungen f�r den Cartographer, darunter die Zuordnung von Frame-Namen, die Verwendung von Sensordaten, die Definition von Trajektorien und die Konfiguration des SLAM-Prozesses. Hier werden Parameter wie die Anzahl der Laser-Scans, die L�nge von fehlenden Datenstrahlen, die Verwendung von IMU-Daten und die Scan-Matching-Parameter festgelegt.

\paragraph{Ausf�hren der ROS-Launch-Datei}

Sobald die Konfigurationsdateien und die ROS-Launch-Datei vorbereitet sind, kann der Google Cartographer gestartet werden. 
Beim start ist folgende reihenfolge zu beachten.

\lstset{style=Konsole}
\begin{lstlisting}[language=bash, caption={Konsole 1:}, 
label={code:example}]
cd catkin_ws
source install_isolated/setup.bash #Sollte eigentlich im bash script automatisch gemacht werden
roscore #Startet den ROS dienst, dieser ist unter anderem f�r den Nachrichtenaustausch verantwortlich
\end{lstlisting}
\begin{lstlisting}[language=bash, caption={Konsole 2:}, 
label={code:example}]
cd Documents\GitHub\Athena-LattePanda-Code\python #Startet den IMU publisher
python3 publisher.py
\end{lstlisting}
\begin{lstlisting}[language=bash, caption={Konsole 3:}, 
label={code:example}]
cd catkin_ws
source install_isolated/setup.bash #Sollte eigentlich im bash script automatisch gemacht werden
roscore														#Startet den ROS dienst, dieser ist unter anderem f�r den Nachrichtenaustausch verantwortlich
roslaunch cartographer_ros athena.launch	#Startet den Cartographer dienst
\end{lstlisting}
\begin{lstlisting}[language=bash, caption={Konsole 4:}, 
label={code:example}]
rosrun rviz rviz	#Startet rviz zur visualisierung des Systems und der Karte
\end{lstlisting}



