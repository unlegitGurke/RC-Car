\fancyfoot[C]{G�rel}
\section{�berlegung und Ansatz}
Im Rahmen der Recherche f�r das Projekt haben sich sehr schnell einige Ideen herauskristallisiert. Ein LiDAR Sensor kann seine Umgebung kartographieren \cite{Kohlbrecher2011} und mit einer daf�r speziell trainierten KI Personen erkennen \cite{Jia2021}. Die generellen Ziele vom Fahrzeug sind daher folgende:
\begin{itemize}
    \item Kartographierung der Schule
    \item Personenerkennung
    \item Routenplanung
    \item Ein System zur Kontrolle des Fahrzeugs, welches die geplante Route abf�hrt.
\end{itemize}

\paragraph{Auswahl Hardware}
Bei der Auswahl der Hardware wurden verschiedene LiDAR-Sensoren und Computer evaluiert, da sich die Studie zur Personenerkennung mit einem LiDAR als sehr versprechend zeigte \cite{Jia2021}. 

\paragraph{Auswahl LiDAR}
Der YDLiDAR G2 ist aufgrund seiner 12Hz Messfrequenz und erweiterten Messdistanz bei moderatem Preis vorteilhaft f�r dieses Projekt \cite{YDLiDAR2024}. 

\paragraph{Auswahl Rechner}
In Bezug auf den Computer stehen der LattePanda 3 Delta und das NVIDIA Jetson Nano zur Auswahl. Der LattePanda ist aufgrund seiner geringeren Kosten und der bew�hrten Zuverl�ssigkeit als "klassischer" x86-basierter Computer bevorzugt \cite{Delta2023} \cite{NVIDIA2024}. Um den LattePanda zu unterst�tzen, wird eine Google Coral TPU mit einem Adapter angeschlossen, diese wird dann f�r notwendige Berechnungen von z.B. der Personen Erkennung KI benutzt \cite{Coral2024}. Die Verwendung der Coral TPU Bibliothek schr�nkt das System auf eine Python Version <3.7 ein.

\paragraph{Verwenden der Sensoren}
F�r die Implementierung ist die ROS (Robot Operating System) Plattform, die vielversprechendste L�sung. ROS bietet eine umfassende und flexible Plattform f�r die Entwicklung von Robotersystemen, die eine nahtlose Integration von Sensordaten und KI-Anwendungen erm�glicht \cite{SAIL2018}. Das ROS System erf�llt die Anforderungen des Projekts am besten und stellt eine solide Grundlage f�r die Implementierung der geplanten Funktionen dar. Da bei einem Projekt mit den verschiedensten Sensoren und Ausschlaggebend f�r die Entscheidung ist daf�r die Kartographier Bibliothek von Google, da dieses nur limitierte Eing�nge ben�tigt und am meisten ausgereift ist. Die Bibliothek gibt dementsprechend folgende Versionen vor:
\begin{itemize}
    \item Ubuntu 20.04
    \item ROS 1
    \item Python <3.8
\end{itemize}

\paragraph{Auswahl ROS und Ubuntu}
Die erste Version von ROS ist f�r dieses Projekt mehr als passend, da durch das Alter die Version etablierter ist. Dementsprechend gibt es mehr Dokumentation und Bibliotheken f�r diese. Ausschlaggebend f�r die Versionswahl sind die SLAM-Bibliotheken, die meist f�r ROS 1 geschrieben sind sowie die LiDAR Bibliothek, welche exklusiv auf ROS 1 funktioniert. Die Auswahl von ROS schreibt dementsprechend auch die Ubuntu Version 20.04 (Codename: �Focal Fossa�).
